\subsection{Assumptions}

The following section contains an outline of the assumptions that have been made during the development of the product, assumptions that the prospective user should keep in mind.

When players have only one card in their individual decks and the outcome of the current round is a draw, then they lose the game as their cards go to the communal deck of cards. This means that, if the game has only two players left and the above scenario happens, then the winner ends up having less than the total of 40 cards in hand.

Another assumption is that in the case of a draw, the next stage of the game is considered to be a new round and not an extension of the draw round. Last winner player will still choose the category, but the game is between all of the players that are still in the game i.e. the ones that still have cards in their decks, and not only those that had the same category values.

Is should also be noted that an AI player will always select the category that holds the best value on his top most card, and the selection will not be random. That will enhance the realistic feel of the game.

It can also be assumed that after every round the content of the communal deck is being collected by the winner player. This means that for the program, it doesn't matter if the result of a round was a draw or not. If the pile is empty and there is a winner, all players submit their cards in the empty pile and the winner player gets the cards from the round back. If the pile is non empty (i.e. the previous round resulted to a draw), then the winner still gets the current cards and the ones that were previously submitted because of the draw.

Another point that should not surprise a potential gamer, is that, if a player has only one card in hand and wins the current round, then his winning card will still be on top for the next round to come. That happens because the winner collects the cards from the other players and puts them at the back of his deck. Since he, had only one card to play with, this card remains on top by default. Again, that will only happen in the case of one card in hand and a winning result for this card holder. 

The online mode assumes that once the player has pressed the button for selecting a category, he cannot change his mind and chose another one. That is, his option has been saved and the only available next step is to show the outcome of the round.

In command line, after the player has selected the category he wants to play with, his choice is only saved after pressing ``enter'', so he has the chance to change the category before selecting to proceed to the next step.

Another point is, that, as soon as the human player has been eliminated, he has no interest in watching the rest of the game, and therefore, in both modes, there is an option to fast- forward to the end, for which is notified.

Finally, the number of AI players in the online mode is determined by adjusting the JSON file.
