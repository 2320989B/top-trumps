
\newpage

\section{Testing}

The tests have been derived from the user stories of section 2.\\
\vspace{0.1cm}\\
The user persona Bob, wants to play a Top Trumps game in the command line.    
\vspace{0.4cm}\\
\begin{tabular}{l | p{10cm}}
User Story & \textbf{S0010} \\
\hline
Type of Test & Initiate the game from command line\\
\hline
Used Variable & 
 - "java -jar ITSD2018Project-1.0.jar -c" \\
 & 
 - "java -jar ITSD2018Project-1.0.jar -c -t"
\\
\hline
Result & [Passed] The program returns the main menu to the player, (figure 1 Appendix). As shown above, both game options were used as variables to test the initialisation of the game.\\ 
\hline
Anti-Variables & - "java -jar ITSD2018Project-1.0.jar -command"\\
& - "java -jar ITSD2018Project-1.0.jar -c -o"\\
& - "java -jar ITSD2018Project-1.0.jar -n"...\\
\hline
Results & [Exception] The program never initiates the game, if the flag is anything else excepts "-c", "-c -t" or "-o". The program throws an exception in case where the user tries to operate both modes, command line mode and on-line mode. In all other cases the program shows the greeting message and terminates.\\
\hline 
\end{tabular}\\
\vspace{0.8cm}\\ 
The user persona Bob can now select one of the three options of the game.\\
\vspace{0.2cm}\\
\begin{tabular}{l | p{10cm}}
User Story & \textbf{S0010}\\ \hline
Type of Test & Select an integer number from the provided options of the main menu.\\ \hline
Used Variables & - integer number, where n=[1,3].\\ \hline
Result & [Passed] The player enter the integer n=1, which initiates a new game. The program returned with a new game.\\ \hline
Anti-Variables & - String\\
& - Character\\
& - Double or Integer, where n!=[1,3]\\
\hline
Results & [Exception] The program throws an exception when the player does not enter the specified integer between n = [1,2,3]. In case the player enters a different integer, the program throws a message indicating the exact integer variable, where in all other situations the program prints out a message demanding an integer.\\ \hline 
\end{tabular}\\
\vspace{0.8cm}\\
After the proper response of Bob to the game commands, he excepts from the game to load the cards, shuffle them, and divide to all players as equally as possible.\\
\vspace{0.2cm}\\
\begin{tabular}{l | p{10cm}}
User Story & \textbf{S0020}\\ \hline
Type of Test & Examination of the test log file for the correct loading of deck, correct shuffling and dividing\\ \hline
Used Variables & No variables for this test.\\ \hline
Result & [Passed] The program does the deck loading, shuffling, creating players' deck and choose randomly the active player correctly. \\ \hline
Anti-Variables & \textit{No anti-variables for this test} \\ \hline
Results & \textit{[No Exception] There is no results for this test.} \\ \hline
\end{tabular}\\
\vspace{0.8cm}\\
Bob wants to have a detailed round and understand who is active player, which category is selected and who won the round.\\
\vspace{0.2cm}\\
\begin{tabular}{l | p{10cm}}
User Story & \textbf{S0030} \& \textbf{S0040} \\ \hline
Type of Test & Game round quality test.\\ \hline
Used Variables & - Press Enter button to show winner\\ \hline
Result & [Passed] The program shows the round number, who is active player, all the cards that participate in this specific round, the selected category inside square brackets and the round winner.\\ \hline
Anti-Variables & \textit{No anti-variables for this quality test} \\ \hline
Results & [\textit{No Exception]} there is no exception for this test, only through the test log file. \\ \hline
\end{tabular}\\
\vspace{0.8cm}\\
Bob, who is an expert user of command line, wants to have a record of the entire game in details in a test log file. He is able to do so, with the -t flag when he initiate the game\\
\vspace{0.2cm}\\
\begin{tabular}{l | p{10cm}}
User Story & \textbf{S0050}\\ \hline
Type of Test & Run the game with the "-t" flag.\\ \hline
Used Variables & - "java -jar ITSD2018Project-1.0.jar -c -t"\\ \hline
Result & [Passed] The program recognise the -t flag and checks if there is no file as "toptrumps.txt" log file, it creates one, or it rewrites in the existing one.\\ \hline
Anti-Variables & - "java -jar ITSD2018Project-1.0.jar -c -*", where * is anything else than "-t".\\ \hline
Results & \textit{[No Exception]} The program recognise the -c flag but it does not produce or write in the "toptrumps.txt" file. \\ \hline
\end{tabular}\\
\vspace{0.8cm}\\
Bob is very cautious about the validity of the game. As a result he wants to examine who won the round and if the AI player selected the highest attribute.\\
\vspace{0.2cm}\\
\begin{tabular}{l | p{10cm}}
User Story & \textbf{S0130} \\ \hline
Type of Test & Print players top cards. Quality test \\ \hline
Used Variables & \textit{No variables used to this test}\\ \hline
Result & [Passed] The program prints out the top cards from all available players. Moreover, the program shows the selected category.\\ \hline
Anti-Variables & \textit{No anti-variables used for this test.}\\ \hline
Results & \textit{[No Exception] There is no exception for this test.}\\ \hline
\end{tabular}\\
\vspace{0.8cm}\\
Bob is the active player in the game and wants to select a category. The program provides him with the option to enter a integer number that matches a specific category.\\
\vspace{0.2cm}\\
\begin{tabular}{l | p{10cm}}
User Story & \textbf{S0030}\\ \hline
Type of Test & Human Player is active player and selects a category\\ \hline
Used Variables & Integer number n=[1,5]\\ \hline
Result & [Passed] The program receives the input integer variable and makes the desired category active. Then, it returns to the system out the results of the round.\\ \hline
Anti-Variables & String, Characters, Double or Integer variables except n=[1,5].\\ \hline
Results & [Exception] The program throws an exception when the human player did not enter one of the five options. The program catches NULL values, strings, characters, doubles and integer number except the n=[1,5]\\ \hline
\end{tabular}\\
\vspace{0.8cm}\\
Bob continues his game and he managed to eliminate some of the players, but he wants to know who has been eliminated.\\
\vspace{0.2cm}\\
\begin{tabular}{l | p{10cm}}
User Story & \textbf{S00--}\\ \hline
Type of Test & Show who players have been eliminated while the human player is in the game. Quality test.\\ \hline
Used Variables & \textit{No variables used for this test.}\\ \hline
Result & [Passed] The program prints out a list with all the active players of the game in each round, right above the players hand.\\ \hline
Anti-Variables & \textit{No anti-variables used for this test.}\\ \hline
Results & \textit{[No exception] There is no exception for this test.}\\ \hline
\end{tabular}

